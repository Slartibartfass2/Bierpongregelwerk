\documentclass[a4paper,11pt]{scrartcl}

\usepackage[T1]{fontenc}
\usepackage[utf8]{inputenc} %Zeichencodierung
\usepackage[ngerman]{babel} %Sprachpaket
\usepackage{setspace} % Einstellungen für den Zeilenabstand
\usepackage{microtype} %typografische Verbesserungen
\usepackage[colorlinks=false,pdfborder={0 0 0},bookmarksnumbered]{hyperref}
\usepackage{enumitem}
\usepackage[usenames,dvipsnames]{xcolor}
\usepackage{tabularx}
\usepackage{lmodern}

%%%%%%%%%%%%%%%%%%%%%%%%%%%%%%%
\newcommand{\enum}[1]{\begin{enumerate}[label=(\arabic*)]#1\end{enumerate}}
\newcommand{\art}[2]{\subsection*{#1} \enum{#2}}
\newcommand{\quot}[1]{\glqq #1\grqq}

\newcommand{\new}[1]{\textcolor{red}{#1}}

\newcounter{art}
\setcounter{art}{1}

\newcommand{\entry}[4]{#1 & #2 & #3 & #4\\}
\newcommand{\event}[4]{\item #1:\\1. #2\\2. #3\\3. #4}
%%%%%%%%%%%%%%%%%%%%%%%%%%%%%%%

\title{\Huge{Bierpongregelwerk}}
\author{Millane Meyer, Markus Schuster, Moritz Gehring \\Leonhard Alkewitz, Mukhtar Muse, Moritz Wieland \\\tiny{Die Pioniere}\\Anpassung durch Felix Schlegel}
\date{\small{\today}}

\begin{document}

\maketitle
\vspace*{-1cm}
\center{\small{Version 2.0.0}}
\newpage

\section{Allgemeines}
    \art{Art \theart}{
        \item
            Wird ein Becher getroffen, so muss \emph{jedes} Teammitglied, die vor Spielbeginn vereinbarte Menge, trinken.
    }
    \stepcounter{art}

    \art{Art \theart}{
        \item
            Es muss erst getrunken werden, wenn alle Würfe des gegnerischen Teams getätigt wurden.
        \item
            Vergisst ein Teammitglied bis zur nächsten Runde zu trinken, so darf sich das gegnerische Team eine angemessene Strafe für das \emph{ganze} Team ausdenken. Empfohlen wird eine Shotrunde.
        \item
            Befindet sich das Spiel am Ende und es handelt sich um einen Konterwurf, so muss erst getrunken werden, wenn der Konterwurf nicht getroffen wurde.
    }
    \stepcounter{art}

    \art{Art \theart}{
        \item
            Bei jedem Spezialwurf wird \emph{immer} der Spezialbecher entfernt.
        \item
            Bei Spezialwürfen, bei denen mehr als nur der getroffene Becher entfernt wird, werden zuerst angrenzende Becher entfernt. Erst danach dürfen Becher entfernt werden, die nicht direkt and den Spezialbecher angrenzen.
        \item
            Die Becher, die zusätzlich entfernt werden müssen, dürfen vom Team ausgesucht werden, denen die Becher gehören.
    }
    \stepcounter{art}

    \art{Art \theart}{
        \item
            Es dürfen beide Bälle gleichzeitig geworfen werden.
    }
    \stepcounter{art}

    \art{Art \theart}{
        \item
            Ein Spieler darf das andere Team Auffordern die Becher zu richten. Das gegnerische Team muss dieser Aufforderung nachkommen und die Becher richten.
        \item
            Das gegnerische Team darf die Becher nicht ohne Erlaubnis des anderen Teams richten.
    }
    \stepcounter{art}

    \art{Art \theart}{
        \item
            Ein Spieler darf das andere Team Auffordern den Ball, wenn er sich in einem Becher befindet, zu entfernen. Das gegnerische Team darf dies auch ohne Aufforderung tun.
    }
    \stepcounter{art}

\section{Spielbeginn}
    \art{Art \theart}{
        \item
            Wurf um Spielreihenfolge und -seite: Je ein Spieler pro Seite erhält einen Ball. Beide Spieler müssen nun gleichzeitig versuchen einen gegnerischen Becher zu treffen.
        \item
            Zum Zeitpunkt des Werfens müssen sich beide Spieler in die Augen schauen.
        \item
            Das Team des Spielers, der zuerst getroffen hat, darf entscheiden welches Team anfängt. Das Team des Spielers, das nicht getroffen hat, darf entscheiden welches Team auf welcher Tischseite spielt.
        \item
            Treffen beide Spieler, so muss der Wurf von den gleichen Spielern wiederholt werden.
    }
    \stepcounter{art}

\section{Spielende}
    \art{Art \theart}{
        \item
            Das Spiel ist vorbei sobald ein Team keine Becher mehr hat.
        \item
            Wird eine Partie mit einem Timer gespielt, so ist das Spiel auch vorbei, sobald der Timer abgelaufen ist. Befindet sich das Spiel zu diesem Zeitpunkt im Gleichstand, so gewinnt das Team, welches den nächsten gegnerischen Becher trifft. Es gibt keinen Konter.
    }
    \stepcounter{art}

    \art{Art \theart}{
        \item
            Wenn das erste Team den letzten Becher vom zweiten Team trifft, hat das zweite Team einen Konverversuch. Ist der Konterversuch erfolgreich werden keine Becher entfernt und das erste Team ist an der Reihe.
        \item
            Ein Konterversuch ist erfolgreich wenn das Zweite Team mit der gleichen Anzahl an Würfen die gleiche Anzahl an Bechern trifft wie das Erste Team zum Ausmachen benötigt hat.
        \item
            Ist der letzte Wurf vom Ersten Team ein Spezialwurf, so muss das Zweite Team mit dem gleichen Spezialwurf kontern.
        \item
            Es gibt keine Begrenzung, wie oft gekontert werden kann.
    }
    \stepcounter{art}

\section{Spielregeln}
    \art{Art \theart}{
        \item
            Jedes Team hat pro Wurfrunde zwei Standardwürfe.
        \item
            Durch Spezialwürfe kann die Anzahl an Würfen pro Wurfrunde erhöht werden.
        \item
            Besteht ein Team aus mehr als einem Spieler, so müssen immer zwei verschiedene Spieler werfen. Sind drei oder mehr Spieler in einem Team, so muss jede Wurfrunde ein anderes Paar an Spielern werfen.
    }
    \stepcounter{art}

    \art{Art \theart\ - Ellenbogen}{
        \item
            Der Ellenbogen muss beim Wurf hinter der Tischkante bleiben. Wird dies nicht eingehalten, so zählt der Wurf nicht.
        \item
            \textbf{Millane-Technik}: Steht eine Person seitlich vom Tisch, so muss der Ellenbogen ebenfalls hinter der verlängerten Tischkante bleiben.
        \item
            Die \textbf{Millane-Technik} darf nur angewendet werden, wenn dadurch kein anderer Spieler behindert wird.
    }
    \stepcounter{art}

    \art{Art \theart\ - \quot{Balls Back}}{
        \item
            Wird in einer Wurfrunde ein Wurf getroffen, so wird der Ball an das aktuelle Werferteam zurückgegeben und der Spieler darf erneut werfen.
        \item
            Bevor erneut geworfen wird, müssen beide Spieler ihre vorherigen Würfe abgeschlossen haben und alle getroffenen Becher entfernt werden.
    }
    \stepcounter{art}

    \art{Art \theart\ - Aufsetzer}{
        \item
            Wird ein Becher mit einem Ball getroffen, der mindestens einen Aufsetzer gemacht hat, so zählt dieser Treffer als zwei Treffer.
        \item
            Sobald ein Ball mindestens einmal aufgesetzt ist, darf dieser abgewehrt werden.
        \item
            Ein Ball zählt als Aufsetzer, sobald er mindestens einmal auf dem Tisch, oder einem Gegenstand auf dem Tisch, aufgesetzt hat.
    }
    \stepcounter{art}

    \art{Art \theart\ - Roller}{
        \item
            Roller zählen als Aufsetzer.
        \item
            Ein Ball zählt als Roller, sobald er mindestens 2 Becher berührt hat.
        \item
            Ein Ball, welcher nur einen Becherrand berührt, bevor er in einen Becher geht, zählt nicht als Roller.
    }
    \stepcounter{art}

    \art{Art \theart}{
        \item
            Schmeißt ein Teammitglied, während des Spiels, einen oder mehrere Becher des eigenen Teams um, so muss ein Becher entfernt werden.
        \item
            Der letzte Becher darf nicht durch umschmeißen entfernt werden.
    }
    \stepcounter{art}

    \art{Art \theart\ - \quot{Blasen}}{
        \item
            Es ist erlaubt einen Ball, welcher noch nicht komplett im Becher ist, durch \quot{Blasen} zu retten.
        \item
            Ein Ball ist noch nicht komplett im Becher, wenn er die Flüssigkeit im Becher noch nicht berührt hat.
        \item
            Wenn der Ball aus einem Becher in einen anderen geblasen wird, zählen beide Becher als getroffen und müssen entfernt werden.
    }
    \stepcounter{art}

    \art{Art \theart\ - \quot{Todesbecher}}{
        \item
            Bleibt ein getroffener Becher am Ende einer Runde stehen, so zählt dieser Becher als \quot{Todesbecher}. Wird dieser Becher von einem gegnerischen Team in einer der nachfolgenden Runden getroffen, so hat das Team, das diesen Becher getroffen hat \emph{sofort} gewonnen.
        \item
            Bemerkt ein Mitglied des Teams mit dem \quot{Todesbecher}, dass sie einen \quot{Todesbecher} haben, so darf dieser umgehend entfernt werden.
    }
    \stepcounter{art}

    \art{Art \theart\ - \quot{Trickshot}}{
        \item
            Rollt ein Ball nach einem Wurf auf die eigene Seite des Tisches zurück, also über die Mitte des Tisches, so darf sich dieser Ball vom aktuell werfenden Team geholt werden. Der Spieler, der sich den Ball geholt hat darf einen \quot{Trickshot} machen. Dieser muss nach den regulären Würfen geworfen werden. Zudem werden von regulären Bällen getroffene Becher zuerst entfernt.
        \item
            Ein Ball gilt nur als zurückgerollt, wenn er auf der eigenen Tischhälfte liegt. Rollt er wieder in die gegnerische Hälfte zurück, darf sich der Ball nicht mehr geholt werden.
        \item
            Becher die von einem \quot{Trickshot} getroffen wurden, müssen \emph{sofort} entfernt werden.
        \item
            Rollt ein \quot{Trickshot}-Ball auf die eigene Tischhälfte zurück, so kann dieser ebenfalls geholt werden und ein weiterer \quot{Trickshot} gemacht werden.
        \item
            Ein \quot{Trickshot} zählt nicht als \quot{Air Ball}, falls dieser hinter der Tischkante gefangen wird.
    }
    \stepcounter{art}

    \art{Art \theart\ - \quot{Bombe}}{
        \item
            Treffen beide Würfe eines Teams in einer Wurfrunde den gleichen Becher, müssen dieser Becher und zwei weitere Becher entfernt werden.
    }
    \stepcounter{art}

    \art{Art \theart\ - \quot{Insel}}{
        \item
            Steht ein Becher allein, hat also keine Nachbarn, so kann ein Spieler \quot{Insel} auf diesen Becher rufen. Trifft der Spieler den genannten Becher muss der getroffene und ein weiterer Becher entfernt werden. Trifft der Spieler statt dem \quot{Insel}-Becher einen anderen Becher, zählt der Treffer nicht.
        \item
            Jedes Team darf zweimal im ganzen Spiel \quot{Insel} rufen. Besteht das Team aus zwei oder mehreren Spielern, müssen es unterschiedliche Spieler sein, welche den \quot{Insel}-Wurf ausführen.
    }
    \stepcounter{art}

    \art{Art \theart\ - \quot{Air Ball}}{
        \item 
            Ein Ball, welcher, ohne den Tisch oder die Becher darauf zu berühren, über die Tischkante des gegnerischen Teams fliegt, als \quot{Air Ball}.
        \item
            Wird ein \quot{Air Ball} hinter der Tischkante gefangen, darf der Fänger des Balls (solange er im gegnerischen Team ist) in der nächsten Wurfrunde einen Ball doppelt werfen.
        \item
            Bälle, die seitlich, neben den Tisch geworfen werden bzw. dort gefangen werden, zählen nicht als \quot{Air Ball}.
    }
    \stepcounter{art}

\section{Spezialwurfkombinationen}
    \begin{itemize}
        \item
            Bombe mit Aufsetzer $\rightarrow$ Bombenbecher + 2 weitere (Bombe) + 1 Becher je Aufsetzer
        \item
            Island mit Aufsetzer $\rightarrow$ Islandbecher + 1 weiterer (Island) + 1 Becher je Aufsetzer
        \item
            Bombe mit Island $\rightarrow$ Bomben-/ Islandbecher + 2 weitere (Bombe) + 1 weiterer (Island)
    \end{itemize}

\end{document}
